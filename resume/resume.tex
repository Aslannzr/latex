\documentclass[letterpaper,final]{moderncv}
\usepackage{hyperref}
\usepackage{color}

\usepackage{fontspec}

% moderncv themes
\moderncvtheme[blue]{casual}                % idem

% DOCUMENT LAYOUT
\usepackage[scale=0.85]{geometry}
\setlength{\hintscolumnwidth}{2cm} % if you want to change the width of the column with the dates
%\AtBeginDocument{\setlength{\maketitlenamewidth}{6cm}}  % only for the classic theme, if you want to change the width of your name pl
\AtBeginDocument{\recomputelengths} % required when changes are made to page layout lengths

% HYPERLINK SETUP

\hypersetup{
  bookmarks=true,         % show bookmarks bar?
  pdftoolbar=true,        % show Acrobat’s toolbar?
  pdfmenubar=true,        % show Acrobat’s menu?
  pdffitwindow=false,     % window fit to page when opened
  pdftitle={Rémy Léone | Resume},    % title
  pdfauthor={Rémy Léone},     % author
  pdfsubject={Rémy Léone | Resume},   % subject of the document
  pdfcreator={Rémy Léone},   % creator of the document
  pdfproducer={Rémy Léone}, % producer of the document
  pdfkeywords={cv} {resume} {Rémy Léone}, % list of keywords
  pdfnewwindow=true,      % links in new window
  colorlinks=true,       % false: boxed links; true: colored links
  linkcolor=red,          % color of internal links
  citecolor=green,        % color of links to bibliography
  filecolor=magenta,      % color of file links
  urlcolor=blue           % color of external links
}

% CHARACTER ENCODING

\defaultfontfeatures{Mapping=tex-text}

% FONTS

\setromanfont [Ligatures={Common},Numbers={OldStyle}]{Gentium Basic}
\setmonofont[Scale=0.8]{Inconsolata}
\setsansfont[Scale=0.9]{Droid Sans}

% personal data
\firstname{Rémy}
\familyname{Leone}
\title{Rémy Léone's resume | PhD thesis in computer-science from September 2012}
\address{Appartement B008 - 61 Boulevard de l'Yerres}{91000 Evry}
\mobile{06 67 56 40 22}
%\homepage{sieben.fr}
\phone{01 60 78 87 07}
\email{remy.leone@etu.upmc.fr}
%\photo[64pt]{picture} % '64pt' is the height the picture must be resized to and 'picture' is the name of the picture file; optional, remove the line if not wanted
\extrainfo{22 ans}
\nopagenumbers{}                             % uncomment to suppress automatic page numbering for CVs longer than one page


%----------------------------------------------------------------------------------
%            content
%----------------------------------------------------------------------------------
\begin{document}
\maketitle

\section{Education}

\cventry{Sept 2011 to present}{\href{upmc.fr}{UPMC}}
{Université Pierre et Marie Curie}{Paris}{}
{Networks with an emphasis on security, Quality of Service, Carrier-Grade networks.
  \newline
\textit{Electives:} Graphs, Distributed Resilience to Attacks, Advanced Security.}

\cventry{Sept 2009 to present}{\href{ensiie.fr}{ENSIIE}}
{Ecole Nationale Supérieure d'Informatique pour l'Industrie et l'Entreprise}{Evry}{}
{Design techniques, modeling, programming, combined with applied mathematics, business management, economics,
  project management, ensuring good versatility.
  \newline
  \textit{Electives:} Operating Systems, Local Area Networks Administration, Security,
Concurrent Programming, Networks Quality of Service.}

\section{Professional Experience}

\cventry{April 2012 to September 2012}{Master's internship}
{
  \href{www.thalesgroup.com}{Thales Communications \& Security}
}
{Thales, Colombes}
{}{
  \begin{itemize}
    \item State of the art of the IETF draft related to Wireless Sensors Networks (CORE \& ROLL).
    \item Building up interface between \href{www.contiki-os.org}{Contiki} and \href{zeromq.org}{ZeroMQ}.
  \end{itemize}
  \textit{Technical environment:} Java, \href{www.contiki-os.org}{Contiki},
  Cooja, \href{https://people.inf.ethz.ch/mkovatsc/californium.php}{Californium},
  \href{zeromq.org}{ZeroMQ}
}

\cventry{June 2011 to September 2011}{Software developer}
{\href{www.paris-montagne.org}{Paris-Montagne}}
{ENS Ulm, Paris}{}{
  \begin{itemize}
    \item Programmed a geolocation tool in order to provide statistical information
      about geographic repartition of students.
    \item Used free software and a free geographic licence in order
      to respect privacy of the users (Python, Django, PostGIS, Leaflet, OpenStreetMap).
  \end{itemize}
  \textit{Technical environment:} Django, Python, GNU/Linux Debian, PostgreSQL,
PostGis, HTML, CSS, Javascript, Leaflet, OpenStreetMap, git, jQuery, nginx.}

\cventry{June 2010 to September 2010}{Scientific code refactoring}
{\href{http://www-dam.cea.fr/}{CEA DAM}}
{Bruyères-le-Châtel}{}{
  Refactored of a Fortran nuclear physics application from a Sun operating system to a Red Hat Entreprise Linux.
  \newline
\textit{Technical environment:} gfortran, Sun OS, Red Hat Entreprise Linux, Shell.}

\section{Computer Skills \& Languages}

\cvcomputer{Databases:}{MySQL, PostgreSQL, PostGIS}
{OS:}{Windows (XP, Vista, 7), GNU/Linux, MacOSX}
\cvcomputer{Open Formats:}{HTML, CSS, YAML, JSON, XML, \LaTeX, Xe\LaTeX}
{Coding tools:}{Eclipse, vim, git, Subversion}
\cvcomputer{Languages:}{Python, Javascript, Shell, C, C++, Java, Ocaml, Assembly, Fortran}
{Frameworks:}{Django, geoDjango, jQuery}

\cvlanguage{French}{Native Speaker}{}
\cvlanguage{English}{TOEIC 855/990}{}
\cvlanguage{Spanish}{Working knowledge}{}

\section{Associative Experience}

\cventry{Every summer since 2006}{Volunteer position}
{\href{http://www.paris-montagne.org/}{Paris-Montagne}}
{\href{http://www.ens.fr/}{ENS Ulm - Paris}}{}
{The purpose of this association is to show how the world
  of scientific research works. Our target audience predominently
  consists of high school students to whom we offer educational activities.
  I was running a stand during a scientific festival
giving pratical information on high technology at \href{http://www.ens.fr/}{ENS Ulm}.}

\cventry{2009}{Scientific Presenter}
{\href{http://www.jardin-experimental.com/}{Jardin Experimental}}
{\href{http://www.cite-sciences.fr/fr/cite-des-sciences/}{Cité des Sciences et de l’Industrie - Paris}}{}
{I was running a stand about the new technologies in la Cité des Sciences et
  de l’Industrie. The Cité des Sciences et de l’Industrie is the biggest science
  museum in Europe. Located in Parc de la Villette in Paris, France, it is at
  the heart of the Cultural Center of Science, Technology and Industry (CCSTI), a
center promoting science and science culture.}

\cventry{Summer 2007}{International research internship}{Summer School of Science}{Višnjan - Croatia}{}
{Scientific research in English, dealing with the computational chemistry
of a Cubane molecule. This experience introduced me to foreign
scientists and helped me finetune my English by giving presentations.}

\end{document}

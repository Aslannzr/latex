\documentclass[NF,paper=a4,pagesize,enlargefirstpage=on,fontsize=12pt,sieben]{scrlttr2}

\LoadLetterOption{NF}

\usepackage{fontspec}
\usepackage{xunicode}
\usepackage{xltxtra}
\usepackage{polyglossia}

\setlength{\footskip}{0pt}

\setdefaultlanguage{french}

\setmainfont[Mapping=tex-text]{Gentium Basic}

\KOMAoptions{pagenumber=off,backaddress=false}

\begin{document}


\begin{letter}{
	Responsable de l'EFU d'Informatique \\
		Case courrier 166 \\
		Maison de la Pédagogie, couloir B, 2e étage, porte 214 \\
		4 place Jussieu \\
		75252 PARIS CEDEX 05}


		\setkomavar{subject}{Candidature Master 2 d'Informatique, spécialité Réseaux}
		\opening{Madame, Monsieur,}

% Introduction

Actuellement en seconde année à l'École Nationale Supérieure d'Informatique
pour l'Industrie et l'Entreprise (ENSIIE) à Évry, je souhaite intégrer
l'UPMC afin de m'y inscrire en Master 2 d'Informatique, spécialité Réseaux.

% Pourquoi je veux faire ce master

L'orientation de votre master vers la recherche scientifique sur les réseaux,
leur sécurité et leur supervision est quelque chose que je désire.  En effet,
après ce master, je souhaite continuer mes études vers un doctorat et rentrer
dans la recherche en informatique sur les réseaux informatiques, leur
fonctionnement et leur amélioration.


Cette volonté d'orienter mon projet professionnel vers la recherche est le
fruit d'une réflexion mature que j'ai menée depuis l'âge de 16 ans,
correspondant à mon entrée dans l'association Paris Montagne et au sein de son
programme phare, la Science Académie. Ce programme a pour but de faire
découvrir les métiers de la recherche à de jeunes lycéens, il a fait naître en
moi une volonté de faire de la recherche, ce qui ne s'est pas arrêté depuis. 

Convaincu que la formation orientée recherche de votre master ainsi que sa
renommée à l'étranger est celle qui me convient pour continuer mon projet
professionnel vers la recherche; je vous propose ma candidature.


\closing{Je vous prie d'agréer Madame, Monsieur, l'expression de mes sincères
salutations.}

\end{letter}
\end{document}

\documentclass[sommairechap,stylejchiquet]{these_gi}
 
\begin{document}
 
% ==================================================================
% OPTIONS D'AFFICHAGE
% non-d�finitif (soumis aux rapporteurs) ou  d�finitif
\definitiftrue
% \definitiffalse
 
% ==================================================================
% RENSEIGNEMENTS SUR LA TH�SE
 
\titleFR{Le titre en fran�ais}
\titleEN{Le titre en anglais}
\abstractFR{Le r�sum� en fran�ais ($\approx$ 1000 caract�res)}
\abstractEN{Le r�sum� en anglais ($\approx$ 1000 caract�res)}
\keywordsFR{Les mots-cl�s en fran�ais}
\keywordsEN{Les mots-cl�s en anglais}
 
\author{nom de l'auteur}
\address{adresse email}
\universite{nom de l'universit�}
\laboratoire{nom du laboratoire}
\specialite{sp�cialit� de la th�se}
\datesoutenance{la date de soutenance}
\datesoumission{la date de soumission aux rapporteurs}
\jury{\begin{tabular}{llll}
    M\up{me} & \textsc{Lesley Truc} & University machin & (Rapporteur) \\
    M. & \textsc{Robert Mitchum} & Laboratoire bidule & (Rapporteur) \\
    M. & \textsc{John Robert} & UTC & (Directeur) \\
       & etc. &  \\
  \end{tabular}    
}
 
% ==================================================================
% D�DICACE
\dedicate{� qui vous voulez\dots}
 
% ==================================================================
% DEBUT DE LA PR�FACE
\beforepreface
 
% remerciements
\chapter*{Remerciements}

%% Encadrant
\malettrine{J}{e}  voudrais tout  d'abord exprimer  mes  plus profonds
remerciements �\dots AH���H !

% �lo�se
Je conclurai en  remerciant de tout c{\oe}ur (l'�tre aim�).

\vspace{2cm}

\hfill Lieu, le \today.


 
% table des mati�res g�n�rale
\tableofcontents
 
% affiche la liste des figures
\listoffigures
 
% ==================================================================
\afterpreface
 
% ==================================================================
% AVANT-PROPOS
\chapter*{Pr�face, Introduction\dots}
\addcontentsline{toc}{chapter}{Pr�face}

% Introduction au contexte
\malettrine{D}{ans}  les  milieux industriels  comme  \dots\\*

% Le sujet de la th�se
L'objectif de cette th�se a  �t� de \dots\\*

% ANNONCE DU PLAN DE LA TH�SE
Nos contributions portent sur : \dots \\*

Le \emph{premier chapitre} expose  la probl�matique de la th�se.

Le \emph{deuxi�me chapitre} pr�sente  en d�tail le mod�le utilis�.\\*

etc.

Cette th�se  a fait l'objet de  divers travaux �crits : \dots


\adjustmtc
 
% ==================================================================
% CONTENU G�N�RAL
\chapter{Introduction}

\begin{chapintro}
  \malettrine{C}{e}  chapitre  introductif  gnagnagna.

  Pas obligatoire !

\end{chapintro}

%%% --------------------------
%%% SECTION 1 --- CONTEXTE ---

\section{Contexte}

\subsection{Enjeux et motivations}

Ce sont les motivations

\subsection{Objectifs et approche g�n�rale}

Ce sont lesobjectifs

%%% ---------------------------------------------------
%%% SECTION 3 --- SYST�MES DYNAMIQUES STOCHASTIQUES ---
\section{Syst�mes diff�rentiels stochastiques et d�gradation}

O� l'on commence � dire de vrais choses...

Exemple de r�f�rences : dans le texte \citet{articlealbert62}, entre parenth�se \citep{articleanisimov77}

On utilise les commande \verb:\citet: et \verb:\citep: du paquet \verb:NatBib:

%%% ----------------------------------------------------------------------
%%% CONCLUSION CHAPITRE UN
\section*{Conclusion du chapitre}
\addcontentsline{toc}{section}{Conclusion}

Ceci est la conclusion. Personnellement, je n'aime pas que la conclusion 
soit num�rot�, mais je veux qu'elle apparaisse dans la table des mati�re, d'o� 
la commande addcontentsline.


\chapter{Le mod�le}

\begin{chapintro}
  \malettrine{D}{ans} ce chapitre, nous pr�sentons le mod�le\dots
\end{chapintro}

%%% ----------------------------------------
%%% SECTION 1 --- CONSTRUCTION DU MOD�LE ---

\section{Construction du syst�me dynamique}

Dans cette  section, nous construisons  la classe de Processus d'int�r�t.
La  construction s'op�re par  morceaux successifs sur les trajectoires 
des processus.

Les   trajectoires  du  processus   peuvent  repr�senter,   selon  les
applications, l'�tat d'une  population de particules (en neutronique),
d'une population  de bact�ries  (en biologie), la  concentration d'une
prot�ine dans une  solution (en chimie) ou, dans  notre cas, l'�tat du
niveau  de d�gradation  d'une structure  (la taille  d'une  fissure se
propageant dans une structure).

Auparavant,  nous avons  besoin d'introduire  quelques  d�finitions et
notations  concernant  les  processus   de  Markov  et  les  processus
markoviens de saut.

\subsection{Rappels sur les processus de Markov}

Nous d�finissons un processus de Markov de la mani�re suivante :

\begin{definitionf}\label{def:processus_markov}
  Soit  $(\Omega,\mathcal{F},\mathbb{P})$  un  espace  probabilis� et  soit  $(X_t,t\in
  \mathbb{R}_+)$  un  processus al�atoire  �  valeurs  dans  un espace  d'�tat
  mesurable   $E$  de   tribu  $\mathcal{E}$.    Notons  $\mathcal{F}_t$   la  tribu
  d'�v�nements   engendr�e    par   $(X_s,0\leq   s    \leq   t)$   et
  $(\mathcal{F}_t)_{t\in\mathbb{R}_+}$ la filtration associ�e.

  Le processus  $X_t$ est  un {\em processus  de Markov} si  pour tout
  $B\in \mathcal{E}$ et pour  tout $s,t \in \mathbb{R}_+$ tels que $0  \leq s < t$,
  il satisfait
  \begin{equation*}
    \mathbb{P}(X_t \in B | \mathcal{F}_s) = \mathbb{P}(X_t \in B | X_s), \qquad p.s.
  \end{equation*}
  De plus, $X_t$ est {\em homog�ne  par rapport au temps} si pour tout
  $t,s\in \mathbb{R}_+$ et pour tout $x\in E$, alors
  \begin{equation}
    \label{eq:c2_def_homogene}
    \mathbb{P}(X_t\in B | X_0=x) = \mathbb{P}(X_{t+s}\in B | X_{s}=x).
  \end{equation}
  Pour  un processus  de Markov  homog�ne, nous  notons  $P(x,B,t)$ la
  probabilit� \eqref{eq:c2_def_homogene}.  La fonction d�finie par $P:
  (x,B,t) \rightarrow P(x,B,t)$ pour $x\in E, B\in\mathcal{E},t\in\mathbb{R}_+$ est
  appel�e \emph{fonction de transition} du processus.
\end{definitionf}

La   figure  \ref{fig:c2_markov}   repr�sente  une   trajectoire  d'un
processus markovien de saut, avec les notations associ�es.

\figScale{c2_markov}{Trajectoire type d'un processus de saut}

%%% --------------------------
%%% CONCLUSION DU CHAPITRE ---

\section*{Conclusion du chapitre}
\addcontentsline{toc}{section}{Conclusion}

Encore une\dots



 
% ==================================================================
% CONCLUSION
\chapter*{Conclusion g�n�rale}
\addcontentsline{toc}{chapter}{Conclusion g�n�rale}

Enfin : la conclusion g�n�rale !!!

Au cours  de ce m�moire, nous  avons d�velopp� un  mod�le \dots

\begin{enumerate}
\item \textbf{Mod�lisation}

\item \textbf{Inf�rence statistique}

\end{enumerate}

\section*{Perspectives}

Dans la  continuit� directe  de notre travail  de th�se,  nous pouvons
\dots


 
% ==================================================================
% ANNEXES
\appendix
\section{Preuve du th�or�me truc}

Ce  th�or�me  est  un  r�sultat  classique  donn�,  par  exemple,  par\dots

 
% ==================================================================
% BIBLIOGRAPHIE
\bibliography{biblio}
 
% ==================================================================
% NOTATIONS
\chapter*{Notations}
\addcontentsline{toc}{chapter}{Notations}

\pagestyle{plain}

\begin{supertabular}{ll}
  PMDM & Processus de Markov d�terministe par morceaux \\
  $p.s.$ & presque s�rement \\
  $\mathbb{N},\mathbb{N}^*$ & ensemble des entiers naturels, des entiers strictement positifs\\
  $\mathbb{R},\mathbb{R}_+$ & ensembles des r�els et des r�els positifs \\ 
  $\mathbb{R}^d$   & ensemble des vecteurs r�els � $d$ dimensions \\
  $\mathbb{P},\mathbb{E}$ & probabilit� et esp�rance \\
  $(\Omega,\mathcal{F},\P)$  & espace probabilis� \\
  $B_t$ & mouvement brownien \\
  $|E|$ & cardinal de l'ensemble $E$ \\
\end{supertabular}

\chapterend

 
% ==================================================================
% COLOPHON
\colophon{Ce document a �t� pr�par� � l'aide de l'�diteur de texte GNU
  Emacs et du logiciel de composition typographique \LaTeXe.}
 
% ==================================================================
% COUVERTURE : RESUME ET MOTS-CL�S
\abstractpage
 
\end{document}
%%% Local Variables:
%%% mode: latex
%%% TeX-master: t
%%% End:
